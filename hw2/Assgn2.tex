\documentclass{article}
\linespread{1.3}
\usepackage[margin=50pt]{geometry}
\usepackage{amsmath, amsthm, amssymb, amsthm, tikz, fancyhdr}
\pagestyle{fancy}
\renewcommand{\headrulewidth}{0pt}
\newcommand{\changefont}{\fontsize{15}{15}\selectfont}

\fancypagestyle{firstpageheader}
{
  \fancyhead[R]{\changefont Michael Huang \\ CFRM 415 \\ Homework 2}
}

\begin{document}

\thispagestyle{firstpageheader}

\section*{1.}
{\Large 

Given a portfolio of three securities with covariances of returns in a matrix $\sigma$, where the sum of the portfolio weights $\omega_1, \omega_2, \omega_3$ is 1, show that the matrix form for the gradient of the Lagrangian for the Markowitz global minimum variance portfolio object function \\
$\frac{1}{2}\omega^T \sigma \omega$ \\
is \\
$\sigma\omega + \lambda1 = 0$ \\
where $\omega$ is the vector containing the portfolio weights. \\ \\ 

We use the optimization

\framebox[1.1\width]{\textbf{answer}}

}

\section*{2.}
{\Large

Given the result from problem 1 is true for any number of assets in a portfolio, and the constraint \\ 
$\omega^T1 = 1$ \\
show that the optimal weights for the global minimum variance portfolio are \\ 
$\omega^* = \frac{\sigma^{-1}1}{1^T\sigma^{-1}1}$ \\
That is, show this using matrix algebra, and not individual elements as done in problem 1. You may find the identity $(AB)^T = B^TA^T$ useful, for two square $k \times k$ matrices $A$ and $B$. \\ \\

Interesting

\framebox[1.1\width]{\textbf{answer}}

}

\section*{3.}
{\Large 

You are given the market term structure of interest rates as of today, date $t$, and you are about to place a trade for a receiver interest rate swap that will have its first reset date and time $T_0 > t$, with both fixed and floating rates paid at successive dates $T_1, T_2, \dots, T_n$, where the floating rate to be paid at each date $T_i$ is fixed on the previous date $T_{i-1}$. The fixed rate to be paid is $S$, $\tau_i = \tau(T_{i-1}, T_i)$ is the dat count-adjusted year fraction on each interval, and the forward interest rate over each interval as seen today is represented by $F(t; T_{i-1}, T_i)$, as we discussed in class.

\subsection*{(a)}

Write the expression for the present value of the stream of fixed interest payments as of today.

\framebox[1.1\width]{\textbf{answer}}

\subsection*{(b)}

Write the expression for the present value of the stream of floating interest payments as of today.

\framebox[1.1\width]{\textbf{answer}}

\subsection*{(c)}

Show that for this swap to be priced fairly, we must have \\
$S = \frac{P(t, T_0) - P(t, T_n)}{\sigma_{i=1}^{n}\tau_iP(t, T_i)}$ \\ 
if the forward rate is represented by the simply-compounded spot interest rate; i.e., \\ 
$F(t; T_{i-1}, T_i) = L(T_{i-1}, T_i) = \frac{1-P(T_{i-1}, T_i)}{\tau_iP(T_{i-1}, T_i)}$ \\ 
Hint: Apply the computation method for a forward discount factor as shown at the top of Slide 10 in the 003 deck to $P(T_{i-1}, T_{i}))$ \\ \\ 

\framebox[1.1\width]{\textbf{answer}}

}

\end{document}