\documentclass{article}
\linespread{1.3}
\usepackage[margin=50pt]{geometry}
\usepackage{amsmath, amsthm, amssymb, amsthm, tikz, fancyhdr}
\pagestyle{fancy}
\renewcommand{\headrulewidth}{0pt}
\newcommand{\changefont}{\fontsize{15}{15}\selectfont}

\fancypagestyle{firstpageheader}
{
  \fancyhead[R]{\changefont Michael Huang \\ CFRM 415 \\ Homework 2}
}

\begin{document}

\thispagestyle{firstpageheader}

\section*{1.}
{\Large 

We optimize using \\
$\underset{\omega_1, \omega_2, \omega_3}{\text{min}} \frac{1}{2}\omega^T \Sigma \omega$ \\
s.t. $\omega_1 + \omega_2 + \omega_3 = 1$, or $\omega^T 1 = 1$ \\ \\
We know that to the Langrangian will be in the form \\
$\mathcal{L} (\omega, \lambda) = \frac{1}{2}\omega^T \Sigma \omega - \lambda (\omega^T 1 - 1)$ \\ \\
% for which we take the gradient to get \\
% $\nabla_\omega \mathcal{L} (\omega, \lambda) = \Sigma \omega + \lambda 1 = 0$
In matrix form, we can depict the first term as follows: \\
$\frac{1}{2}$
$\cdot$
$\begin{bmatrix}
  \omega_1 & \omega_2 & \omega_3
\end{bmatrix}$ 
$\cdot$
$\begin{bmatrix}
  \sigma_1^2 & \sigma_{12} & \sigma_{13} \\
  \sigma_{12} & \sigma_2^2 & \sigma_{23} \\
  \sigma_{13} & \sigma_{23} & \sigma_3^2 \\
\end{bmatrix}$
$\cdot$
$\begin{bmatrix}
  \omega_1 \\
  \omega_2 \\
  \omega_3 \\
\end{bmatrix}$ \\
$= \frac{1}{2}$
$\cdot$
$\begin{bmatrix}
  \omega_1 \sigma_1^2 + \omega_2 \sigma_{12} + \omega_3 \sigma_{13} & \omega_1 \sigma_{12} + \omega_2 \sigma_2^2 + \omega_3 \sigma_{23} & \omega_1 \sigma_{13} + \omega_2 \sigma_{23} + \omega_3 \sigma_3^2
\end{bmatrix}$
$\cdot$
$\begin{bmatrix}
  \omega_1 \\
  \omega_2 \\
  \omega_3 \\
\end{bmatrix}$ \\
$= \frac{1}{2}$
$\cdot$
$\begin{bmatrix}
\omega_1^2 \sigma_1^2 
  + \omega_2^2 \sigma_2^2 
  + \omega_3^2 \sigma_3^2 
  + 2 \omega_1 \omega_2 \sigma_{12} 
  + 2 \omega_1 \omega_3 \sigma_{13} 
  + 2 \omega_2 \omega_3 \sigma_{23}
\end{bmatrix}$ \\ 
and the latter term as follows: \\
$\lambda \cdot ($
$\begin{bmatrix}
  \omega_1 & \omega_2 & \omega_3
\end{bmatrix}$
$\cdot$ 
$\begin{bmatrix}
  1 \\
  1 \\
  1 \\
\end{bmatrix}$
$- 1)$ \\ \\

We take the gradient of the entire thing: \\
$\nabla \omega [\frac{1}{2}$
$\cdot$
$\begin{bmatrix}
\omega_1^2 \sigma_1^2 
  + \omega_2^2 \sigma_2^2 
  + \omega_3^2 \sigma_3^2 
  + 2 \omega_1 \omega_2 \sigma_{12} 
  + 2 \omega_1 \omega_3 \sigma_{13} 
  + 2 \omega_2 \omega_3 \sigma_{23})
\end{bmatrix}$
$+ \lambda (\omega^T1 - 1]$ \\ 
$= \frac{1}{2}$
$\cdot$
$\begin{bmatrix}
2 \omega_1 \sigma_1^2 
  + 2 \omega_2 \sigma_2^2 
  + 2 \omega_3 \sigma_3^2 
  + 2 \omega_1 \sigma_{12} 
  + 2 \omega_2 \sigma_{12} 
  + 2 \omega_1 \sigma_{13} 
  + 2 \omega_3 \sigma_{13} 
  + 2 \omega_2 \sigma_{23}
  + 2 \omega_3 \sigma_{23}
\end{bmatrix}$
$+ \lambda^T1$ \\ 
$= 
\begin{bmatrix}
  \omega_1 \sigma_1^2 
  + \omega_2 \sigma_2^2 
  + \omega_3 \sigma_3^2 
  + \omega_1 \sigma_{12} 
  + \omega_2 \sigma_{12} 
  + \omega_1 \sigma_{13} 
  + \omega_3 \sigma_{13} 
  + \omega_2 \sigma_{23}
  + \omega_3 \sigma_{23}
\end{bmatrix}$ 
$+ \lambda^T1$ \\ \\

More simply: \\
$0 = \nabla_{\omega} [\frac{1}{2}\omega^T \Sigma \omega - \lambda (\omega^T 1 - 1)]$ \\
$0 = \frac{1}{2}(\Sigma \omega + \omega^T \Sigma) - \lambda 1$ \\
$0 = \frac{1}{2}(\Sigma \omega + \Sigma\omega) - \lambda 1$ \\
$0 = \frac{1}{2}(2 \Sigma \omega) - \lambda 1$ \\
$0 = \Sigma \omega - \lambda 1$

}

\section*{2.}
{\Large

Given the result from problem 1 is true for any number of assets in a portfolio, and the constraint \\ 
$\omega^T1 = 1$ \\
show that the optimal weights for the global minimum variance portfolio are \\ 
$\omega^* = \frac{\Sigma^{-1}1}{1^T\Sigma^{-1}1}$ \\
That is, show this using matrix algebra, and not individual elements as done in problem 1. You may find the identity $(AB)^T = B^TA^T$ useful, for two square $k \times k$ matrices $A$ and $B$. \\ \\

We found in problem 1 that the matrix form is $\Sigma \omega + \lambda 1 = 0$. In other words, \\
$\Sigma \omega = -\lambda1$ \\
$\omega = -\lambda\Sigma^{-1}1$ \\ \\
Our previous constraint was that $\omega^T 1 = 1$, which we can now re-form as \\
$(-\lambda \Sigma^{-1} 1)^T 1 = 1$ \\
$-\lambda 1^T {\Sigma^{-1}}^T 1 = 1$ \\
$-\lambda = \frac{1}{1^T {\Sigma^{-1}}^T 1}$ \\
$\lambda = -\frac{1}{1^T {\Sigma^{-1}}^T 1}$ \\ \\
And plugging this back in, we can find that \\
$\omega = -\Sigma^{-1}1 \cdot -\frac{1}{1^T {\Sigma^{-1}}^T 1}$ \\
$\omega = -\frac{-\Sigma^{-1}1}{1^T {\Sigma^{-1}}^T 1}$ \\
$\omega = \frac{-Sigma^{-1}1}{1^T {\Sigma^{-1}}^T 1}$ \\
as we hoped to show.

}

\section*{3.}
{\Large 

You are given the market term structure of interest rates as of today, date $t$, and you are about to place a trade for a receiver interest rate swap that will have its first reset date and time $T_0 > t$, with both fixed and floating rates paid at successive dates $T_1, T_2, \dots, T_n$, where the floating rate to be paid at each date $T_i$ is fixed on the previous date $T_{i-1}$. The fixed rate to be paid is $S$, $\tau_i = \tau(T_{i-1}, T_i)$ is the dat count-adjusted year fraction on each interval, and the forward interest rate over each interval as seen today is represented by $F(t; T_{i-1}, T_i)$, as we discussed in class.

\subsection*{(a)}

Write the expression for the present value of the stream of fixed interest payments as of today.

\framebox[1.1\width]{\textbf{answer}}

\subsection*{(b)}

Write the expression for the present value of the stream of floating interest payments as of today.

\framebox[1.1\width]{\textbf{answer}}

\subsection*{(c)}

Show that for this swap to be priced fairly, we must have \\
$S = \frac{P(t, T_0) - P(t, T_n)}{\sigma_{i=1}^{n}\tau_iP(t, T_i)}$ \\ 
if the forward rate is represented by the simply-compounded spot interest rate; i.e., \\ 
$F(t; T_{i-1}, T_i) = L(T_{i-1}, T_i) = \frac{1-P(T_{i-1}, T_i)}{\tau_iP(T_{i-1}, T_i)}$ \\ 
Hint: Apply the computation method for a forward discount factor as shown at the top of Slide 10 in the 003 deck to $P(T_{i-1}, T_{i}))$ \\ \\ 

\framebox[1.1\width]{\textbf{answer}}

}

\end{document}
