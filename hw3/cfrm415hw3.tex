\documentclass{article}
\linespread{1.3}
\usepackage[margin=50pt]{geometry}
\usepackage{amsmath, amsthm, amssymb, amsthm, tikz, fancyhdr}
\pagestyle{fancy}
\renewcommand{\headrulewidth}{0pt}
\newcommand{\changefont}{\fontsize{15}{15}\selectfont}

\fancypagestyle{firstpageheader}
{
  \fancyhead[R]{\changefont Michael Huang \\ CFRM 415 \\ Homework 3}
}

\begin{document}

\thispagestyle{firstpageheader}

\section*{1.}

\framebox[1.1\width]{\textbf{answer}}

{\Large 

Show that for a European call option, we get the stronger result \\
$C \geq S - PV(X) $ \\
Include a diagram to illustrate this.  What does this tell us about the time value of a European call option? \\
Remark:  The result is the same for an American option, but we will discuss this next week.



}

\section*{2.}
{\Large

Suppose a trader writes a European put option with exercise price $X$.  Let $t$ be such that $0 \geq t < T$, where 0 represents the first trading day for the option contract, and $T$ is its expiration date.

\subsection*{(a)}

What does the trader believe will happen to the underlying stock price $S$ when writing the put?

\subsection*{(b)}

Draw the payoff diagram at expiration, let $V$ be the value of the option position in the graph.

\subsection*{(c)}

Draw the curve showing the value of the position for some arbitrary value of $t$.

\subsection*{(d)}

Show a point on the curve (drawn in c)) where the put option is deep in the money.  What can you say about the intrinsic value and the time value?

\subsection*{(e)}

Show a point on the curve (drawn in c)) where the put option is deep out of the money.  What can you say about the intrinsic value and the time value now?

}

\section*{3.}
{\Large 

A European capped call option is like a European call option except that the payoff is $H - X$ for some constant $H > X$, instead of $S - X$, when the terminal stock price exceeds $H$.  Construct a portfolio of European options with an identical payoff. Explain your reasoning.   

}

\end{document}