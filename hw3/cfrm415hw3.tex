\documentclass{article}
\linespread{1.3}
\usepackage[margin=50pt]{geometry}
\usepackage{amsmath, amsthm, amssymb, amsthm, tikz, fancyhdr}
\pagestyle{fancy}
\renewcommand{\headrulewidth}{0pt}
\newcommand{\changefont}{\fontsize{15}{15}\selectfont}

\fancypagestyle{firstpageheader}
{
  \fancyhead[R]{\changefont Michael Huang \\ CFRM 415 \\ Homework 3}
}

\begin{document}

\thispagestyle{firstpageheader}

\section*{1.}

{\Large 

Show that for a European call option, we get the stronger result \\
$C \geq S - PV(X) $ \\
Include a diagram to illustrate this.  What does this tell us about the time value of a European call option? \\ \\

% According to put-call parity, we know \\
% $C - P - S + PV(X) = 0$ \\
% $C - P = S - PV(X)$ \\
% $C = S - PV(X) + P$ \\
% Since $P > 0$, we know that \\
% $C \geq S - PV(X)$

% payoff diagram  

% The time value of a call is more the than the time value of a corresponding put.
% Time value part is greater than just share price or loan?
% Time value is always positive, while time value for put can go negative.

}

\section*{2.}
{\Large

Suppose a trader writes a European put option with exercise price $X$.  Let $t$ be such that $0 \geq t < T$, where 0 represents the first trading day for the option contract, and $T$ is its expiration date.

\subsection*{(a)}

What does the trader believe will happen to the underlying stock price $S$ when writing the put? \\ \\

% The trader believes that the stock price $S$ will go below $X$ at time $T$, once they can exercise the option. Specifically, if the trader pays some premium $P$ on the option, then they would expect the stock price $S$ to go at least $P$ below $X$, or that $S$ will be at most $X - P$ at time $T$. 

\subsection*{(b)}

Draw the payoff diagram at expiration, let $V$ be the value of the option position in the graph. \\ \\

% done? not sure whether or not to include P or nah

\subsection*{(c)}

Draw the curve showing the value of the position for some arbitrary value of $t$. \\ \\

% should curve start below the dotted line?
% should either line go below 0?

\subsection*{(d)}

Show a point on the curve (drawn in c)) where the put option is deep in the money. What can you say about the intrinsic value and the time value? \\ \\

% intrinsic value = equity share price - exercise price
% time value = possibility of becoming more valuable before expiration date

% The intrinsic value will be relatively high (exercise price is low, relative to share price)
% The time value will be relatively low (lower chance of becoming more valuable, depending on t).

\subsection*{(e)}

Show a point on the curve (drawn in c)) where the put option is deep out of the money.  What can you say about the intrinsic value and the time value now? \\ \\

% The intrinsic value will be zero (we cannot exercise the put option for profit, it is worthless)
% The time value will be relatively high (greater chance of becoming more valuable, depends on t)
% could also be smol, depending on diff between option value at/before expiry

}

\section*{3.}
{\Large 

A European capped call option is like a European call option except that the payoff is $H - X$ for some constant $H > X$, instead of $S - X$, when the terminal stock price exceeds $H$.  Construct a portfolio of European options with an identical payoff. Explain your reasoning. \\ \\

% Relative to the original $(C - P - S + PV(X) = 0)$ 

% identical payoff to $H - X$?
% Buy a European (K, t) call option and sell a European (B + K, t) call option. Determine the payoff of this portfolio at time t. Show that this is equivalent to the payoff at time t of the European (K, t) capped call option. 

% check the notebooks for thoughts

}

\end{document}