\documentclass{article}
\linespread{1.3}
\usepackage[margin=50pt]{geometry}
\usepackage{amsmath, amsthm, amssymb, amsthm, tikz, fancyhdr}
\pagestyle{fancy}
\renewcommand{\headrulewidth}{0pt}
\newcommand{\changefont}{\fontsize{15}{15}\selectfont}

\fancypagestyle{firstpageheader}
{
  \fancyhead[R]{\changefont Michael Huang \\ CFRM 415 \\ Midterm}
}

\begin{document}

\thispagestyle{firstpageheader}

\section*{1.}

{\Large 

\framebox[1.1\width]{\textbf{answer}}

\subsection*{(a)}

$\mu^T = 
\begin{bmatrix}
  0.3 & 0.6 & 0.7
\end{bmatrix}$ \\
$ \Sigma = 
\begin{bmatrix}
2.5 & 1.2 & 0.3 \\
1.2 & 3.1 & 1.3 \\
0.3 & 1.3 & 2.4
\end{bmatrix}$ \\ 

We optimize by minimizing $\frac{1}{2} \omega^T \Sigma \omega$ with $\omega^T 1 = 1$, and doing the Lagrangian, we find that \\
$\widehat{\omega} = \frac{\Sigma^{-1}1}{1^T\Sigma^{-1}1}$ \\
$\widehat{\omega} = 
\begin{bmatrix}
  0.33604008 \\ 
  0.04578438 \\ 
  0.34986178 \\ 
\end{bmatrix} 
\div
\begin{bmatrix}
  0.73168625
\end{bmatrix}$ \\
$\widehat{\omega} = 
\begin{bmatrix}
  0.459268 \\ 
  0.06257379 \\
  0.47815821
\end{bmatrix}$ \\
The optimal weights are therefore \framebox[1.1\width]{\textbf{0.459268, 0.06257379, 0.47815821}} \\ \\ 
We simply multiply to get the expected annual return: \\
$= \widehat{\omega}^T \cdot  \mu$ \\
$=
\begin{bmatrix}
  0.51003542
\end{bmatrix}$ \\
The expected annual return is therefore \framebox[1.1\width]{\textbf{0.51003542}} \\ \\
To find the portfolio volatility, we find the square root of the variance: \\
$= \sqrt{\widehat{\omega}^T \Sigma \widehat{\omega}}$ \\
$= \sqrt{\frac{1}{1^T \Sigma^{-1} 1}}$ \\
$= \sqrt{1.36670602}$ \\
$= 1.16906203$ \\
The portfolio volatility is therefore \framebox[1.1\width]{\textbf{1.16906203}}

\subsection*{(b)}

Let $\gamma > 0$ be an investor’s risk tolerance. Calculate the mean and variance of the portfolio returns generated by the weights optimizing the general Markowitz optimization problem. Note: You do not need to compute the optimal weights for this case. \\ \\

% Show previous algebra
We define $A = 1^T \Sigma^{-1} 1$, $B = \mu^T \Sigma^{-1} 1$, $C = \mu^T \Sigma^{-1} \mu$. We can calculate these values as scalars, and redefine them as $A = 0.73168625$, $B = 0.3731859$, $C = 0.23130615$.
We know that the portfolio mean can be represented as \\
$\mu = \frac{B}{A} + \frac{1}{\gamma} \frac{AC - B^2}{A}$ \\
$\mu = \frac{0.3731859}{0.73168625} + \frac{1}{\gamma} \frac{0.73168625 \cdot 0.23130615 - 0.3731859^2}{0.73168625}$ \\
$\mu = 0.51003541477 + \frac{1}{\gamma} \frac{0.16924352949 - 0.13926771595}{0.73168625}$ \\
$\mu = 0.51003541477 + \frac{0.04096812471}{\gamma}$ \\
The portfolio mean is therefore \framebox[1.1\width]{\textbf{$0.51003541477 + \frac{0.04096812471}{\gamma}$}} \\ \\
We know that the portfolio variance can be represented as \\
$\sigma^2 = \frac{1}{A} + \frac{1}{\gamma^2} \frac{AC - B^2}{A}$ \\
$\sigma^2 = \frac{1}{0.73168625} + \frac{1}{\gamma^2} \frac{0.73168625 \cdot 0.23130615 - 0.13926771595}{0.73168625}$ \\
$\sigma^2 = 1.36670601641 + \frac{1}{\gamma^2} \frac{0.16924352949 - 0.13926771595}{0.73168625}$ \\
$\sigma^2 = 1.36670601641 + \frac{0.04096812471}{\gamma^2}$ \\
The portfolio variance is therefore \framebox[1.1\width]{\textbf{$1.36670601641 + \frac{0.04096812471}{\gamma^2}$}}

}

\section*{2.}

{\Large

Let $R_i$ and $R_j$ be random variables representing annual excess returns on securities $i$ and $j$ in the total market, under the assumptions of the CAPM.

\subsection*{(a)}

Write the regression equation for $R_i$, including the statistical conditions on the error terms, and identify the terms representing the systematic and idiosyncratic risks.

\subsection*{(b)}

}

\section*{3.}
{\Large 



}

\section*{4.}
{\Large 



}

\section*{5.}
{\Large 



}


\end{document}