\documentclass{article}
\linespread{1.3}
\usepackage[margin=50pt]{geometry}
\usepackage{amsmath, amsthm, amssymb, amsthm, tikz, fancyhdr}
\pagestyle{fancy}
\renewcommand{\headrulewidth}{0pt}
\newcommand{\changefont}{\fontsize{15}{15}\selectfont}

\fancypagestyle{firstpageheader}
{
  \fancyhead[R]{\changefont Michael Huang \\ CFRM 415 \\ Midterm}
}

\begin{document}

\thispagestyle{firstpageheader}

\section*{1.}

\framebox[1.1\width]{\textbf{answer}}

{\Large 

\subsection*{(a)}

$\mu^T = 
\begin{bmatrix}
  0.3 & 0.6 & 0.7
\end{bmatrix}$ \\
$ \Sigma = 
\begin{bmatrix}
2.5 & 1.2 & 0.3 \\
1.2 & 3.1 & 1.3 \\
0.3 & 1.3 & 2.4
\end{bmatrix}$ \\ 

Compute the optimal weights, the expected annual portfolio return, and the portfolio volatility for the global minimum variance portfolio. You can perform the matrix computations using a tool such as R, Python, Excel etc; in particular, you do not need to write out row reduction. Do be clear, however, in the steps involved in your computation.

\subsection*{(b)}

}

\section*{2.}
{\Large



}

\section*{3.}
{\Large 



}

\section*{4.}
{\Large 



}

\section*{5.}
{\Large 



}


\end{document}