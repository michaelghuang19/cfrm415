\documentclass{article}
\linespread{1.3}
\usepackage[margin=50pt]{geometry}
\usepackage{amsmath, amsthm, amssymb, amsthm, tikz, fancyhdr}
\pagestyle{fancy}
\renewcommand{\headrulewidth}{0pt}
\newcommand{\changefont}{\fontsize{15}{15}\selectfont}

\fancypagestyle{firstpageheader}
{
  \fancyhead[R]{\changefont Michael Huang \\ CFRM 415 \\ Homework 1}
}

\begin{document}

\thispagestyle{firstpageheader}

\section*{1.}
{\Large 

Find the accumulated value of \$100 at the end of 20 years:

\subsection*{(a)}

At an annual interest rate of 6\% simple interest, we use the formula \\
$= A(1 + rt)$ \\
$= 100 \cdot (1 + 0.06 \cdot 20)$ \\ 
$= 100 \cdot (1 + 1.2)$ \\
$= 100 \cdot 2.2$ \\ 
$= $ \framebox[1.1\width]{\textbf{\$220}}

\subsection*{(b)}

At an annual interest rate of 6\% compounded quarterly, we can use the formula \\
$= A \{(1 + \frac{r}{m})^m\}^{\frac{k}{m}}$ \\
$= 100 \cdot \{(1 + \frac{0.06}{4})^4\}^{\frac{4 \cdot 20}{4}}$ \\
$= 100 \cdot \{(1 + 0.015)^4\}^{20}$ \\
$= 100 \cdot \{(1.015)^4\}^{20}$ \\
$= 100 \cdot \{1.06136355062\}^{20}$ \\
$= 100 \cdot \{1.06136355062\}^{20}$ \\
$= 100 \cdot 3.29066278668 $ \\
$= 329.066278668 = $ \framebox[1.1\width]{\textbf{$\sim$\$329.07}}

\subsection*{(c)}

At an annual interest rate of 6\% compounded continuously, we can use the formula \\
$= (Ae^{r})^{\frac{k}{m}} $ \\
$= Ae^{rt} $ \\
$= 100 \cdot e^{0.06 \cdot 20} $ \\
$= 100 \cdot e^{1.2} $ \\
$= 100 \cdot 3.32011692274 $ \\
$= 332.011692274  = $ \framebox[1.1\width]{\textbf{$\sim$\$332.01}}

}

\section*{2.}
{\Large

Find the present value of \$100 to be paid in 20 years

\subsection*{(a)}

At an annual interest rate of 2\% simple interest, we use the formula \\
$PV = \frac{c}{1 + rt}$ \\
$PV = \frac{100}{1 + 0.02 \cdot 20}$ \\
$PV = \frac{100}{1 + 0.4}$ \\
$PV = \frac{100}{1.4}$ \\
$PV = 71.4285714286 = $ \framebox[1.1\width]{\textbf{$\sim$\$71.43}}

\subsection*{(b)}

At an annual interest rate of 2\% compounded semiannually, we use the formula \\
$PV = c \{(1 + \frac{r}{m})^m\}^{\frac{-k}{m}}$ \\
$PV = 100 \{(1 + \frac{0.02}{2})^2\}^{\frac{-(2 \cdot 20)}{2}}$ \\
$PV = 100 \cdot \{(1 + 0.01)^2\}^{-20}$ \\
$PV = 100 \cdot \{(1.01)^2\}^{-20}$ \\
$PV = 100 \cdot \{1.0201\}^{-20}$ \\
$PV = 100 \cdot 0.67165313886$ \\
$PV = 67.165313886$ = \framebox[1.1\width]{\textbf{$\sim$\$67.17}}

\subsection*{(c)}

At an annual interest rate of 2\% compounded continuously, we use the formula \\
$PV = ce^{-rt}$ \\
$PV = 100 \cdot e^{-0.02 \cdot 20}$ \\
$PV = 100 \cdot e^{-0.4}$ \\
$PV = 100 \cdot 0.67032004603$ \\
$PV = 67.032004603 = $ \framebox[1.1\width]{\textbf{$\sim$\$67.03}}

}

\section*{3.}
{\Large 

Find the annual rate of interest at which the accumulated value of \$1000 at the end of 15 years is \$3000

\subsection*{(a)}

If interest is compounded semiannually, we can use the formula and solve for $r$: \\
$AV = A\{(1 + \frac{r}{m})^m\}^{\frac{k}{m}}$ \\
$3000 = 1000 \cdot \{(1 + \frac{r}{2})^2\}^{\frac{2 \cdot 15}{2}}$ \\
$3000 = 1000 \cdot \{(1 + \frac{r}{2})^2\}^{15}$ \\
$3 = \{(1 + \frac{r}{2})^2\}^{15}$ \\
$\sqrt[15]{3} = (1 + \frac{r}{2})^2$ \\
$\sqrt{\sqrt[15]{3}} = 1 + \frac{r}{2}$ \\
$1.03729919731 - 1 = \frac{r}{2}$ \\
$0.03729919731 \cdot 2 = r$ \\
$r = 0.07459839462 = $ \framebox[1.1\width]{\textbf{$\sim$7.46\%}}

\subsection*{(b)}

If interest is compounded continuously, we can use the formula and solve for $r$: \\
$AV = Ae^{rt}$ \\
$3000 = 1000 \cdot e^{r \cdot 15}$ \\
$3 = e^{r \cdot 15}$ \\
$ln(3) = r \cdot 15$ \\
$r = ln(3) \div 15 = 1.09861228867 \div 15 = 0.07324081924 = $ \framebox[1.1\width]{\textbf{$\sim$7.32\%}}

}

\section*{4.}
{\Large 

We calculate the down payment by taking the difference between the present value of the loan payments and the total loan for 10000. In other words, we have that \\
$PV = 10000 = \text{Down Payment} + \text{Monthly Payment Sum}$ \\
We can then solve: \\
$10000 = DP + \frac{250}{\frac{0.18}{12}} \cdot (1 - \frac{1}{(1 + \frac{0.18}{12})^{4 \cdot 12}})$ \\
$10000 = DP + 16666.6666667 \cdot (1 - \frac{1}{(1.015)^{48}})$ \\
$10000 = DP + 16666.6666667 \cdot 0.51063830469$ \\
$DP = 10000 - 8510.63841152 = 1489.36158848 = $ \framebox[1.1\width]{\textbf{$\sim$\$1489.36}}

}

\section*{5.}
{\Large 

A junk bond pays a semiannual coupon at an 8\% annual with 5 years to maturity, with a face value of \$5000. The market yield is 2\%.

\subsection*{(a)}

To calculate the bond price, we use the following formula \\
$P = c \sum_{k=1}^{n} v^k + Fv^n$ \\
$P = c \sum_{k=1}^{n} (1 + \frac{r}{m})^{-k} + Fv^n$ \\
$P = c \sum_{k=1}^{2 \cdot 5} (1 + \frac{0.02}{2})^{-k} + Fv^{2 \cdot 5}$ \\
$P = c \sum_{k=1}^{10} (1.01)^{-k} + Fv^{10}$ \\ 
$P = (5000 \cdot 0.08 \div 2) \cdot 1.01^{-1} \cdot \frac{1 - 1.01^{-10}}{1 - 1.01^{-1}} + F(1.01)^{-10}$ \\
$P = 200 \cdot 0.9900990099 \cdot \frac{1 - 0.90528695469}{1 - 0.9900990099} + 5000 \cdot 0.90528695469$ \\
$P = 200 \cdot 0.9900990099 \cdot \frac{0.09471304531}{0.0099009901} + 5000 \cdot 0.90528695469$ \\
$P = 200 \cdot 0.9900990099 \cdot 9.56601757535 + 5000 \cdot 0.90528695469$ \\
$P = 1894.26090601 + 4526.43477345$ \\
$P = 6420.69567946 = $ \framebox[1.1\width]{\textbf{$\sim$\$6420.70}}

\subsection*{(b)}

To calculate the Macaulay Duration of the bond, we use the following formula: \\
$= \frac{c \sum_{k=1}^{n} (\frac{k}{m}) v^k + (\frac{n}{m})Fv^n}{P}$ \\
$= \frac{(5000 \cdot 0.08 \div 2) \sum_{k=1}^{2 \cdot 5} (\frac{k}{2}) v^k + (\frac{2 \cdot 5}{2})5000v^{2 \cdot 5}}{6420.70}$ \\
$= \frac{200 \sum_{k=1}^{10} (\frac{k}{2}) (1 + \frac{r}{m})^{-k} + 25000\cdot {(1 + \frac{r}{m})}^{-10}}{6420.70}$ \\
$= \frac{200 \sum_{k=1}^{10} (\frac{k}{2}) (1 + \frac{0.02}{2})^{-k} + 25000 \cdot {(1 + \frac{0.02}{2})}^{-10}}{6420.70}$ \\
$= \frac{200 \sum_{k=1}^{10} (\frac{k}{2}) (1.01)^{-k} + 25000 \cdot (1.01)^{-10}}{6420.70}$ \\
$= \frac{200 \sum_{k=1}^{10} (\frac{k}{2}) (1.01)^{-k} + 25000 \cdot (1.01)^{-10}}{6420.70}$ \\
$= \frac{200 \cdot 25.65740 + 25000 \cdot 0.90528695469}{6420.70}$ \\
$= \frac{5131.48 + 22632.1738672}{6420.70}$ \\
$= \frac{27763.6538672}{6420.70} = $ \framebox[1.1\width]{\textbf{$\sim$4.32408520367}}

\subsection*{(c)}

If the market yield jumps to 2.02\%, to determine the approximate change in bond price, we can use the following formula: \\
$\Delta P \cong -P D_M \Delta r$ \\
$\Delta P \cong -P Dv \Delta r$ \\
$\Delta P \cong -6420.70 \cdot 4.32 \cdot (1 + \frac{r}{m})^{-1} \cdot (0.0202 - 0.02)$ \\
$\Delta P \cong -6420.70 \cdot 4.32 \cdot (1 + \frac{0.02}{2})^{-1} \cdot (0.0202 - 0.02)$ \\
$\Delta P \cong -6420.70 \cdot 4.32 \cdot (1.01)^{-1} \cdot (0.0202 - 0.02)$ \\
$\Delta P \cong -6420.70 \cdot 4.32 \cdot 0.9900990099 \cdot 0.0002$ \\
$\Delta P \cong -6420.70 \cdot 4.32 \cdot 0.9900990099 \cdot 0.0002$ \\
$= -5.49255920792 = $\framebox[1.1\width]{\textbf{$\sim$\$-5.49}}

}

\section*{6.}
{\Large 

To calculate the bond price, we use the following formula, and use year-fractions to take into account that there are only effectively 2 payments per year left over the remaining 2 years, and a payment directly 2 months after we buy the bond: \\ 
$P = c \sum_{k=1}^{n} v^k + Fv^n$ \\
$P = (5000 \cdot 0.035 \div 2) \sum_{k=1}^{n} (1 + \frac{r}{m})^{-k} + 5000 \cdot (1 + \frac{r}{m})^{-n}$ \\
We adjust the sum to take into account the year-fraction and semiannual payment (i.e. 2 months = $\frac{1}{3}$ due to semiannual payments rather than $\frac{1}{6}$): \\
$P = 87.5 \cdot \sum_{k=\frac{1}{3}}^{2 \cdot 2 + \frac{1}{3}} (1 + \frac{0.013}{2})^{-k} + 5000 \cdot (1 + \frac{0.013}{2})^{-(2 \cdot 2 + \frac{1}{3})}$  \\
$P = 87.5 \cdot \sum_{k=0}^{4} (1.0065)^{-(k + \frac{1}{3})} + 5000 \cdot (1.0065)^{-\frac{13}{3}}$ \\
$P $ = 87.3112340635 + 86.7473761188 + 86.1871595815 + 85.6305609354 + 85.0775568161 + 4861.57467521 \\ 
$P = 5292.5285627253 = $ \framebox[1.1\width]{\textbf{$\sim$\$5292.53}}


}

\section*{7.}
{\Large 

We aim to show that the modified duration of a continuously compounded zero coupon bond is equal to its time to maturity, that is, $D_M = t$. We know that the modified duration is defined as $D_M = -\frac{1}{P}\frac{dP}{dr} $. \\
For a continuously compounded zero coupon bond, its price is $P(0, t) = Fe^{-rt}$. Plugging this in gives us \\ 
$D_M = -\frac{1}{P}\frac{dP}{dr} (Fe^{-rt})$ \\
$D_M = -\frac{1}{P} \cdot Fe^{-rt} \cdot -t$ \\
$D_M = \frac{Fe^{-rt} \cdot t}{P}$ \\
$D_M = \frac{Fe^{-rt} \cdot t}{Fe^{-rt}}$ \\
$D_M = t$ \\
which is simply the time to maturity, as we aimed to show.

}

\end{document}