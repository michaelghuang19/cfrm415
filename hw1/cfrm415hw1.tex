\documentclass{article}
\linespread{1.3}
\usepackage[margin=50pt]{geometry}
\usepackage{amsmath, amsthm, amssymb, amsthm, tikz, fancyhdr}
\pagestyle{fancy}
\renewcommand{\headrulewidth}{0pt}
\newcommand{\changefont}{\fontsize{15}{15}\selectfont}

\fancypagestyle{firstpageheader}
{
  \fancyhead[R]{\changefont Michael Huang \\ CFRM 415 \\ Homework 1}
}

\begin{document}

\thispagestyle{firstpageheader}

\section*{1.}
{\Large 

Find the accumulated value of \$100 at the end of 20 years:

\subsection*{(a)}

At an annual interest rate of 6\% simple interest, we use the formula \\
$= A(1 + rt)$ \\
$= 100 \cdot (1 + 0.06 \cdot 20)$ \\ 
$= 100 \cdot (1 + 1.2)$ \\
$= 100 \cdot 2.2$ \\ 
$= $ \framebox[1.1\width]{\textbf{\$220}}

\subsection*{(b)}

At an annual interest rate of 6\% compounded quarterly, we can use the formula \\
$= A \{(1 + \frac{r}{m})^m\}^{\frac{k}{m}}$ \\
$= 100 \cdot \{(1 + \frac{0.06}{4})^4\}^{\frac{4 \cdot 20}{4}}$ \\
$= 100 \cdot \{(1 + 0.015)^4\}^{20}$ \\
$= 100 \cdot \{(1.015)^4\}^{20}$ \\
$= 100 \cdot \{1.06136355062\}^{20}$ \\
$= 100 \cdot \{1.06136355062\}^{20}$ \\
$= 100 \cdot 3.29066278668 $ \\
$= 329.066278668 = $ \framebox[1.1\width]{\textbf{$\sim$\$329.07}}

\subsection*{(c)}

At an annual interest rate of 6\% compounded continuously, we can use the formula \\
$= (Ae^{r})^{\frac{k}{m}} $ \\
$= Ae^{rt} $ \\
$= 100 \cdot e^{0.06 \cdot 20} $ \\
$= 100 \cdot e^{1.2} $ \\
$= 100 \cdot 3.32011692274 $ \\
$= 332.011692274  = $ \framebox[1.1\width]{\textbf{$\sim$\$332.01}}

}

\section*{2.}
{\Large

Find the present value of \$100 to be paid in 20 years

\subsection*{(a)}

At an annual interest rate of 2\% simple interest, we use the formula \\
$PV = \frac{c}{1 + rt}$ \\
$PV = \frac{100}{1 + 0.02 \cdot 20}$ \\
$PV = \frac{100}{1 + 0.4}$ \\
$PV = \frac{100}{1.4}$ \\
$PV = 71.4285714286 = $ \framebox[1.1\width]{\textbf{$\sim$\$71.43}}

\subsection*{(b)}

At an annual interest rate of 2\% compounded semiannually, we use the formula \\
$PV = c \{(1 + \frac{r}{m})^m\}^{\frac{-k}{m}}$ \\
$PV = 100 \{(1 + \frac{0.02}{2})^2\}^{\frac{-(2 \cdot 20)}{2}}$ \\
$PV = 100 \cdot \{(1 + 0.01)^2\}^{-20}$ \\
$PV = 100 \cdot \{(1.01)^2\}^{-20}$ \\
$PV = 100 \cdot \{1.0201\}^{-20}$ \\
$PV = 100 \cdot 0.67165313886$ \\
$PV = 67.165313886$ = \framebox[1.1\width]{\textbf{$\sim$\$67.17}}

\subsection*{(c)}

At an annual interest rate of 2\% compounded continuously, we use the formula \\
$PV = ce^{-rt}$ \\
$PV = 100 \cdot e^{-0.02 \cdot 20}$ \\
$PV = 100 \cdot e^{-0.4}$ \\
$PV = 100 \cdot 0.67032004603$ \\
$PV = 67.032004603 = $ \framebox[1.1\width]{\textbf{$\sim$\$67.03}}

}

\section*{3.}
{\Large 

Find the annual rate of interest at which the accumulated value of \$1000 at the end of 15 years is \$3000

\subsection*{(a)}

If interest is compounded semiannually, we can use the formula and solve for $r$: \\
$AV = A\{(1 + \frac{r}{m})^m\}^{\frac{k}{m}}$ \\
$3000 = 1000 \cdot \{(1 + \frac{r}{2})^2\}^{\frac{2 \cdot 15}{2}}$ \\
$3000 = 1000 \cdot \{(1 + \frac{r}{2})^2\}^{15}$ \\
$3 = \{(1 + \frac{r}{2})^2\}^{15}$ \\
$\sqrt[15]{3} = (1 + \frac{r}{2})^2$ \\
$\sqrt{\sqrt[15]{3}} = 1 + \frac{r}{2}$ \\
$1.03729919731 - 1 = \frac{r}{2}$ \\
$0.03729919731 \cdot 2 = r$ \\
$r = 0.07459839462 = $ \framebox[1.1\width]{\textbf{$\sim$7.46\%}}

\subsection*{(b)}

If interest is compounded continuously, we can use the formula and solve for $r$: \\
$AV = Ae^{rt}$ \\
$3000 = 1000 \cdot e^{r \cdot 15}$ \\
$3 = e^{r \cdot 15}$ \\
$ln(3) = r \cdot 15$ \\
$r = ln(3) \div 15 = 1.09861228867 \div 15 = 0.07324081924 = $ \framebox[1.1\width]{\textbf{$\sim$7.32\%}}

}

\section*{4.}
{\Large 

A dealer of questionable repute prices a used car at \$10,000, and it offers a four-year loan at 18\% compounded monthly, where the customer must pay \$250 per month. How much will the down payment be? \\



}

\section*{5.}
{\Large 

A junk bond pays a semiannual coupon at an 8\% annual with 5 years to maturity, with a face value of \$5000. The market yield is 2\%.

\subsection*{(a)}

To calculate the bond price, we use the following formula \\
$P = c \sum_{k=1}^{n} v^k + Fv^n$ \\
$P = c \sum_{k=1}^{n} (1 + \frac{r}{m})^{-k} + Fv^n$ \\
$P = c \sum_{k=1}^{2 \cdot 5} (1 + \frac{0.02}{2})^{-k} + Fv^{2 \cdot 5}$ \\
$P = c \sum_{k=1}^{10} (1.01)^{-k} + Fv^{10}$ \\ 
$P = (5000 \cdot 0.08 \div 2) \cdot 1.01^{-1} \cdot \frac{1 - 1.01^{-10}}{1 - 1.01^{-1}} + F(1.01)^{-10}$ \\
$P = 200 \cdot 0.9900990099 \cdot \frac{1 - 0.90528695469}{1 - 0.9900990099} + 5000 \cdot 0.90528695469$ \\
$P = 200 \cdot 0.9900990099 \cdot \frac{0.09471304531}{0.0099009901} + 5000 \cdot 0.90528695469$ \\
$P = 200 \cdot 0.9900990099 \cdot 9.56601757535 + 5000 \cdot 0.90528695469$
$P = 1894.26090601 + 4526.43477345$ \\
$P = 6420.69567946 = $ \framebox[1.1\width]{\textbf{$\sim$\$6420.70}}

\subsection*{(b)}

To calculate the Macaulay Duration of the bond, we use the following formula: \\
$= $ \framebox[1.1\width]{\textbf{answer}}

\subsection*{(c)}

The market yield jumps to 2.02\%. What is the approximate change in the bond price? \\

}

\section*{6.}
{\Large 

A 10-year bond paying a semiannual coupon of 3.5\% has been issued and is trading in the market. It has a face value of \$5000 and matures in two years and two months. The market yield is 1.3\%. Calculate the bond price. \\

}

\section*{7.}
{\Large 

Show that the modified duration of a continuously compounded zero coupon bond is equal to its time to maturity. \\

}

\end{document}