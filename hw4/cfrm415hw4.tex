\documentclass{article}
\linespread{1.3}
\usepackage[margin=50pt]{geometry}
\usepackage{amsmath, amsthm, amssymb, amsthm, tikz, fancyhdr}
\pagestyle{fancy}
\renewcommand{\headrulewidth}{0pt}
\newcommand{\changefont}{\fontsize{15}{15}\selectfont}

\fancypagestyle{firstpageheader}
{
  \fancyhead[R]{\changefont Michael Huang \\ CFRM 415 \\ Homework 4}
}

\begin{document}

\thispagestyle{firstpageheader}

\section*{1.}
{\Large 

\subsection*{(a)}

Referring to our discussion in class, and on p 94 of Lyuu (9.3), show that \\
$hS + B = \frac{1}{R}[(\frac{R - d}{u - d})C_u + (\frac{u - R}{u - d})C_d] $ \\ \\ 

We know from the replicating portfolio that \\ 
$hSu + RB = C_u$ \\ 
$hSd + RB = C_d$ \\ \\
We solve to find that \\
$hSu + RB - hSd - RB = C_u - C_d$ \\
$hSu - hSd = C_u - C_d$ \\
$hS(u - d) = C_u - C_d$ \\
$hS = \frac{C_u - C_d}{u - d}$ \\ \\
and that \\
$RB = C_u - hSu$ \\
$RB = C_d - hSd$ \\
$2RB = C_u + C_d - (\frac{C_u - C_d}{u - d})(u + d)$ \\ 
$2RB = - (\frac{- uC_u - uC_d + dC_u + dC_d + uC_u - uC_d + dC_u - dC_d}{u - d})$ \\ 
$2RB = - (\frac{2dC_u - 2uC_d}{u - d})$ \\ 
$B = - (\frac{dC_u - uC_d}{(u - d)R})$ \\ 
$B = \frac{uC_d - dC_u}{(u - d)R}$ \\ \\
Putting this all together: \\
$hS + B = \frac{C_u - C_d}{u - d} + \frac{uC_d - dC_u}{(u - d)R}$ \\
$hS + B = \frac{R(C_u - C_d) + uC_d - dC_u}{(u - d)R}$ \\
$hS + B = \frac{R(C_u - C_d) + uC_d - dC_u}{(u - d)R}$ \\
$hS + B = \frac{1}{R}(\frac{RC_u - RC_d + uC_d - dC_u}{(u - d)})$ \\
$hS + B = \frac{1}{R}(\frac{(R-d)C_u + (-R+u)C_d}{(u - d)})$ \\
$hS + B = \frac{1}{R}(\frac{(R-d)C_u}{(u-d)} + \frac{(u-R)C_d}{(u - d)})$ \\
exactly as we sought to show.

\subsection*{(b)}

Assuming $d < R < u$, show that $p = (\frac{R - d}{u - d})$ defines a probability measure. \\ \\

We know that $u > R$, and $d = d$ and is strictly less than both $u$ and $R$. We therefore know that both $R-d$ and $u-d$ are positive, and the difference between these values will simply be the difference between $u$ and $R$. Again, since we know $u > R$, we know that the denominator will always be larger than $R$, which means that the fraction $p$ overall will be some value less than 1 and greater than 0, which means that the value $p$ will define a probability measure. 

}

\section*{2.}
{\Large

You need to price an American put option on a non-dividend paying stock currently worth \$50.  The strike price is also \$50.  The risk-free rate is 10\%, the time to expiration is six months, and the volatility is 40\%.



\subsection*{(a)}

Calculate $u$, $d$, and $p$ for a two-step binomial lattice with $\delta t = 0.25$, using the Jarrow-Rudd scheme for $u$ and $d$. \\ \\

Using Black-Scholes, we have that \\
$u = e^{\sigma \sqrt{\delta t}} = e^{0.4 \cdot 0.5} = $ \framebox[1.1\width]{\textbf{$e^{0.2}$}} \\
$d = u^{-1} = e^{-\sigma \sqrt{\delta t}} = e^{-0.4 \cdot 0.5} = $ \framebox[1.1\width]{\textbf{$e^{-0.2}$}} \\
Using Jarrow-Rudd, we know that \\
$p = \frac{R - d}{u - d} = \frac{e^{r \delta t} - e^{-\sigma \sqrt{\delta t}}}{e^{\sigma \sqrt{\delta t}} - e^{-\sigma \sqrt{\delta t}}} = \frac{e^{0.1 \cdot 0.25} - e^{-0.2}}{e^{0.2} - e^{-0.2}} = $ \framebox[1.1\width]{\textbf{$\frac{e^{0.025} - e^{-0.2}}{e^{0.2} - e^{-0.2}}$}}


\subsection*{(b)}

Value the option using a two-step binomial lattice. \\ \\

We know the final value for a two-step binomial lattice to be \\
$= 2((p \cdot (1-p)) \cdot ud S_0 + u^2 p^2 S_0 + d^2 (1-p)^2 S_0$ \\
$= 2( 0.513 \cdot  0.487) \cdot ( 1.221 \cdot  0.819) S_0 + ( 1.221)^2 \cdot ( 0.513)^2 S_0 + ( 0.819)^2 \cdot ( 0.487)^2 S_0$ \\
$= 2(0.250) \cdot (1.000) S_0 + (1.491) \cdot (0.263) S_0 + (0.671) \cdot (0.237) S_0$ \\
$= 0.500 S_0 + 0.392 S_0 + 0.159 S_0$ \\
$= 1.051 S_0$ \\
$= 52.55$, which is over 50, so \\
\framebox[1.1\width]{\textbf{0}}

\subsection*{(c)}

Is it optimal to exercise this option early at any time on the lattice? If so, where? \\ \\
We have $S = 50$, $X = 50$, $r = 0.1$, $\delta t = 0.25$, $\sigma = 0.4$. \\ \\
The final values are as follows, from top to bot: \\
$u^2 \cdot 50 = 1.491 \cdot 50 = 74.55$, which is a payoff of $0$ \\
$ud \cdot 50 = 50$, which is a payoff of $0$ \\
$d^2 \cdot 50 = 0.671 \cdot 50 = 33.550$, which is a payoff of $16.450$ \\ \\
Our only intermediate values for option values are therefore as follows, from top to bot: \\
$e^{-r \delta t} (p \cdot 0 + (1-p) \cdot 0) = 0$ \\
$e^{-r \delta t} ((1-p) \cdot 16.450) = e^{-0.1 \cdot 0.25} (0.487 \cdot 16.450) = e^{-0.025} = 0.975 * 8.011 = 7.811$. The stock price at this point is $d \cdot 50 = 0.819 \cdot 50 = 40.95$, which when exercised, gives us a value of $50 - 40.95 = 9.05$, so it is indeed optimal to exercise at this point at 3 months when the price decreases to this point.

}

\end{document}